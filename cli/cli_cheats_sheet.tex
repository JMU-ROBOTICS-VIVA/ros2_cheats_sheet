\documentclass[9pt,a4paper]{article}
\usepackage[margin=1cm,landscape]{geometry}
\geometry{
  a4paper,
  left=10mm,
  right=10mm,
  headheight=1cm,
  top=1cm,
  bottom=1.5cm,
  footskip=1cm,
  %showframe
}

%\usepackage{lipsum}

\usepackage{hyperref} % For hyperlinks
\hypersetup{
    colorlinks=true,
    %linkcolor=blue,
    %filecolor=magenta,
    urlcolor=cyan,
}

\usepackage{multicol} % For overall column layout
\usepackage{xcolor} % For colors
\usepackage{tabularx}
\usepackage{fancyhdr} % For footer

\pagenumbering{gobble} % No page numbering
\pagestyle{fancy}
\fancyhead{} % No header
\renewcommand{\headrulewidth}{0pt} % No header rule

\cfoot{\copyright~2019 Canonical} % Actual footer text

% A bunch of command for boilerplate text
\newcommand{\rosverb}[1]{\textbf{\sffamily\color{blue}#1}~~}
\newcommand{\rossubverb}[1]{{\sffamily\color{blue}#1}~~}
\newcommand{\textangles}[1]{\textless #1\textgreater}
\newcommand{\smallhspace}{\-\hspace{0.3cm}}
\newcommand{\terminal}[1]{\-\hspace{0.5cm}{\sffamily\$ #1}}
\newcommand{\terminalinebreak}[1]{\ \textbackslash\hfill\phantom{.}\linebreak\-\hspace{0.5cm}~}
\newcommand{\ddash}{-{}-}

\begin{document}
\setlength\parindent{0pt}
\begin{multicols*}{3}[]

\begin{center}
  \textbf{\Large ROS 2 Cheats Sheet}
\end{center}
%

\hrulefill

\textbf{Command Line Interface}

%
All ROS 2 CLI tools start with the prefix `ros2' followed by a verb,
a sub-verb and (possibly) positional/optional arguments.

For any tool, the documentation is accessible with,\\
%
\terminal{ros2 \rosverb{verb} \ddash help}\\
%
and similarly for sub-verb documentation,\\
%
\terminal{ros2 \rosverb{verb} \rossubverb{sub\_verb} -h}

Similarly, auto-completion is available for all (sub-)verbs and
most positional/optional arguments. E.g.,\\
%
\terminal{ros2 \rosverb{verb} [tab][tab]}

Some of the examples below rely on:\\
\href{https://github.com/ros2/demos}{ROS 2 demos package}.

\hrulefill

%
\rosverb{action} Allows to manually send a goal and
displays debugging information about actions.
\\
Sub-commands:
\\
\begin{tabularx}{\linewidth}{lX}
\smallhspace \rossubverb{info}        & Output information about an action. \\
\smallhspace \rossubverb{list}        & Output a list of action names.      \\
\smallhspace \rossubverb{send\_goal}  & Send an action goal.                \\
\smallhspace \rossubverb{show}        &  Output the action definition.
\end{tabularx}
%
Examples:
\\
\terminal{ros2 action info /fibonacci} \\
\terminal{ros2 action list} \\
\terminal{ros2 action send\_goal /fibonacci
\terminalinebreak~action\_tutorials/action/Fibonacci "{order: 5}"} \\
\terminal{ros2 action show action\_tutorials/action/Fibonacci}

\hrulefill

%
\rosverb{bag} Allows to record/play topics to/from a rosbag.
\\
Sub-commands:
\\
%
\begin{tabularx}{\linewidth}{lX}
\smallhspace \rossubverb{info}      & Output information of a bag.\\
\smallhspace \rossubverb{play}      & Play a bag.\\
\smallhspace \rossubverb{record}    & Record a bag.
\end{tabularx}
%
\\
Examples:
\\
\terminal{ros2 info \textangles{bag-name}}\\
\terminal{ros2 play \textangles{bag-name}}\\
\terminal{ros2 record -a}
%

\hrulefill

%
\rosverb{component} Various component related sub-commands.\\
Sub-commands:
\\
%
\begin{tabularx}{\linewidth}{lX}
\smallhspace \rossubverb{list}          & Output a list of running containers and components.           \\
\smallhspace \rossubverb{load}          & Load a component into a container node.                       \\
\smallhspace \rossubverb{standalone}    & Run a component into its own standalone container node.       \\
\smallhspace \rossubverb{types}         & Output a list of components registered in the ament index.    \\
\smallhspace \rossubverb{unload}        & Unload a component from a container node.
\end{tabularx}
%

\hrulefill

%
\rosverb{daemon} Various daemon related sub-commands.\\
Sub-commands:
\\
\begin{tabularx}{\linewidth}{lX}
\smallhspace \rossubverb{start}  &  Start the daemon if it isn't running.   \\
\smallhspace \rossubverb{status} &  Output the status of the daemon.        \\
\smallhspace \rossubverb{stop}   &  Stop the daemon if it is running
\end{tabularx}
%

\hrulefill

%
\rosverb{extension\_points} List extension points.
%

\hrulefill

%
\rosverb{extensions} List extensions.
%

\hrulefill

%
\rosverb{launch} Allows to run a launch file in an arbitrary package
without to cd there first.
\\
Usage:
\\
\terminal{ros2 launch \textangles{package}~\textangles{launch-file}}
\\
Example:
\\
\terminal{ros2 launch demo\_nodes\_cpp add\_two\_ints.launch.py}
%

\hrulefill

%
\rosverb{lifecycle} Various lifecycle related sub-commands.
\\
Sub-commands:
\\
%
\begin{tabularx}{\linewidth}{lX}
\smallhspace \rossubverb{get}   & Get lifecycle state for one or more nodes.    \\
\smallhspace \rossubverb{list}  & Output a list of available transitions.       \\
\smallhspace \rossubverb{nodes} & Output a list of nodes with lifecycle.        \\
\smallhspace \rossubverb{set}   & Trigger lifecycle state transition.
\end{tabularx}
%

\hrulefill

%
\rosverb{msg} Displays debugging information about messages.
\\
Sub-commands:
\\
\begin{tabularx}{\linewidth}{lX}
\smallhspace \rossubverb{list}      & Output a list of message types.                           \\
\smallhspace \rossubverb{package}   & Output a list of message types within a given package.    \\
\smallhspace \rossubverb{packages}  & Output a list of packages which contain messages.         \\
\smallhspace \rossubverb{show}      & Output the message definition.
\end{tabularx}
%
Examples:
\\
\terminal{ros2 msg list} \\
\terminal{ros2 msg package std\_msgs} \\
\terminal{ros2 msg packages} \\
\terminal{ros2 msg show geometry\_msgs/msg/Pose}
%

\hrulefill

%
\rosverb{multicast} Various multicast related sub-commands.
\\
Sub-commands:
\\
\begin{tabularx}{\linewidth}{lX}
\smallhspace \rossubverb{receive}   &  Receive a single UDP multicast packet. \\
\smallhspace \rossubverb{send}      &  Send a single UDP multicast packet.
\end{tabularx}
%

\hrulefill

%
\rosverb{node} Displays debugging information about nodes.
\\
Sub-commands:
\\
\begin{tabularx}{\linewidth}{lX}
\smallhspace \rossubverb{info}   & Output information about a node. \\
\smallhspace \rossubverb{list}   & Output a list of available nodes.
\end{tabularx}
%
Examples:
\\
\terminal{ros2 node info /talker} \\
\terminal{ros2 node list}
%

\hrulefill

%
\rosverb{param} Allows to manipulate parameters.
\\
Sub-commands:
\\
\begin{tabularx}{\linewidth}{lX}
\smallhspace \rossubverb{delete}    & Delete parameter.                         \\
\smallhspace \rossubverb{get}       & Get parameter.                            \\
\smallhspace \rossubverb{list}      & Output a list of available parameters.    \\
\smallhspace \rossubverb{set}       & Set parameter
\end{tabularx}
%
Examples:
\\
\terminal{ros2 param delete /talker /use\_sim\_time}    \\
\terminal{ros2 param get /talker /use\_sim\_time}       \\
\terminal{ros2 param list}                              \\
\terminal{ros2 param set /talker /use\_sim\_time false}
%

\hrulefill

%
\rosverb{pkg} Create a ros2 package or output package(s)-related information.
\\
Sub-commands:
\\
\begin{tabularx}{\linewidth}{lX}
\smallhspace \rossubverb{create}        &  Create a new ROS2 package.                       \\
\smallhspace \rossubverb{executables}   &  Output a list of package specific executables.   \\
\smallhspace \rossubverb{list}          &  Output a list of available packages.             \\
\smallhspace \rossubverb{prefix}        &  Output the prefix path of a package.             \\
\smallhspace \rossubverb{xml}           &  Output the information contained in the package xml manifest.
\end{tabularx}
%
Examples:
\\
\terminal{ros2 pkg executables demo\_nodes\_cpp}    \\
\terminal{ros2 pkg list}                            \\
\terminal{ros2 pkg prefix std\_msgs}                \\
\terminal{ros2 pkg xml -t version}
%

\hrulefill

%
\rosverb{run} Allows to run an executable in an arbitrary package
without having to cd there first.
\\
Usage:
\\
\terminal{ros2 run \textangles{package}~\textangles{executable}}
\\
Example:
\\
\terminal{ros2 run demo\_node\_cpp talker}
%

\hrulefill

%
\rosverb{security} Various security related sub-commands.
\\
Sub-commands:
\\
%
\begin{tabularx}{\linewidth}{lX}
\smallhspace \rossubverb{create\_key}            & Create key.                                  \\
\smallhspace \rossubverb{create\_permission}     & Create keystore.                             \\
\smallhspace \rossubverb{generate\_artifacts}    & Create permission.                           \\
\smallhspace \rossubverb{list\_keys}             & Distribute key.                              \\
\smallhspace \rossubverb{create\_keystore}       & Generate keys and permission
files from a list of identities and policy files.                                               \\
\smallhspace \rossubverb{distribute\_key}        & Generate XML policy file from ROS graph data.\\
\smallhspace \rossubverb{generate\_policy}       & List keys.
\end{tabularx}
%
Examples (see \href{https://github.com/ros2/sros2}{sros2 package}):
\\
\terminal{ros2 security create\_key demo\_keys /talker}   \\
\terminal{ros2 security create\_permission demo\_keys /talker
\terminalinebreak~policies/sample\_policy.xml} \\
\terminal{ros2 security generate\_artifacts} \\
%\terminal{ros2 security list\_keys} \\
\terminal{ros2 security create\_keystore demo\_keys}
%\terminal{ros2 security distribute\_key} \\
%\terminal{ros2 security generate\_policy}

\hrulefill

%
\rosverb{service} Allows to manually call a service and displays debugging
information about services.
\\
Sub-commands:
\\
\begin{tabularx}{\linewidth}{lX}
\smallhspace \rossubverb{call}   & Call a service.                            \\
\smallhspace \rossubverb{find}   & Output a list of services of a given type. \\
\smallhspace \rossubverb{list}   & Output a list of service names.            \\
\smallhspace \rossubverb{type}   & Output service's type.
\end{tabularx}
%
Examples:
\\
\terminal{ros2 service call /add\_two\_ints
\terminalinebreak~example\_interfaces/AddTwoInts "{a: 1, b: 2}"}  \\
\terminal{ros2 service find rcl\_interfaces/srv/ListParameters}   \\
\terminal{ros2 service list}                                      \\
\terminal{ros2 service type /talker/describe\_parameters}
%

\hrulefill

%
\rosverb{srv} Various srv related sub-commands.
\\
Sub-commands:
\\
%
\begin{tabularx}{\linewidth}{lX}
\smallhspace \rossubverb{list}      & Output a list of available service types.\\
\smallhspace \rossubverb{package}   & Output a list of available service types within one package.\\
\smallhspace \rossubverb{packages}  & Output a list of packages which contain services.\\
\smallhspace \rossubverb{show}      & Output the service definition.
\end{tabularx}
%

\hrulefill

%
\rosverb{test} Run a ROS2 launch test.
%

\hrulefill

%
\rosverb{topic} A tool for displaying debug information about ROS topics,
including publishers, subscribers, publishing rate, and
messages.
\\
Sub-commands:
\\
%
\begin{tabularx}{\linewidth}{lX}
\smallhspace \rossubverb{bw}    & Display bandwidth used by topic.                  \\
\smallhspace \rossubverb{delay} & Display delay of topic from timestamp in header.  \\
\smallhspace \rossubverb{echo}  & Output messages of a given topic to screen.       \\
\smallhspace \rossubverb{find}  & Find topics of a given type type.                 \\
\smallhspace \rossubverb{hz}    & Display publishing rate of topic.                 \\
\smallhspace \rossubverb{info}  & Output information about a given topic.           \\
\smallhspace \rossubverb{list}  & Output list of active topics.                     \\
\smallhspace \rossubverb{pub}   & Publish data to a topic.                          \\
\smallhspace \rossubverb{type}  & Output topic's type.
\end{tabularx}
%
Examples:
\\
\terminal{ros2 topic bw /chatter}                         \\
%\terminal{ros2 topic delay} \\
\terminal{ros2 topic echo /chatter}                       \\
\terminal{ros2 topic find rcl\_interfaces/msg/Log}        \\
\terminal{ros2 topic hz /chatter}                         \\
\terminal{ros2 topic info /chatter}                       \\
\terminal{ros2 topic list}                                \\
\terminal{ros2 topic pub /chatter std\_msgs/msg/String
\terminalinebreak~'data: Hello ROS 2 world'}              \\
\terminal{ros2 topic type /rosout}                        \\

\end{multicols*}
\end{document}
